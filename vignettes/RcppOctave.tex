\documentclass[english,10pt,a4paper]{article}\usepackage[]{graphicx}\usepackage[]{color}
%% maxwidth is the original width if it is less than linewidth
%% otherwise use linewidth (to make sure the graphics do not exceed the margin)
\makeatletter
\def\maxwidth{ %
  \ifdim\Gin@nat@width>\linewidth
    \linewidth
  \else
    \Gin@nat@width
  \fi
}
\makeatother

\definecolor{fgcolor}{rgb}{0.345, 0.345, 0.345}
\newcommand{\hlnum}[1]{\textcolor[rgb]{0.686,0.059,0.569}{#1}}%
\newcommand{\hlstr}[1]{\textcolor[rgb]{0.192,0.494,0.8}{#1}}%
\newcommand{\hlcom}[1]{\textcolor[rgb]{0.678,0.584,0.686}{\textit{#1}}}%
\newcommand{\hlopt}[1]{\textcolor[rgb]{0,0,0}{#1}}%
\newcommand{\hlstd}[1]{\textcolor[rgb]{0.345,0.345,0.345}{#1}}%
\newcommand{\hlkwa}[1]{\textcolor[rgb]{0.161,0.373,0.58}{\textbf{#1}}}%
\newcommand{\hlkwb}[1]{\textcolor[rgb]{0.69,0.353,0.396}{#1}}%
\newcommand{\hlkwc}[1]{\textcolor[rgb]{0.333,0.667,0.333}{#1}}%
\newcommand{\hlkwd}[1]{\textcolor[rgb]{0.737,0.353,0.396}{\textbf{#1}}}%

\usepackage{framed}
\makeatletter
\newenvironment{kframe}{%
 \def\at@end@of@kframe{}%
 \ifinner\ifhmode%
  \def\at@end@of@kframe{\end{minipage}}%
  \begin{minipage}{\columnwidth}%
 \fi\fi%
 \def\FrameCommand##1{\hskip\@totalleftmargin \hskip-\fboxsep
 \colorbox{shadecolor}{##1}\hskip-\fboxsep
     % There is no \\@totalrightmargin, so:
     \hskip-\linewidth \hskip-\@totalleftmargin \hskip\columnwidth}%
 \MakeFramed {\advance\hsize-\width
   \@totalleftmargin\z@ \linewidth\hsize
   \@setminipage}}%
 {\par\unskip\endMakeFramed%
 \at@end@of@kframe}
\makeatother

\definecolor{shadecolor}{rgb}{.97, .97, .97}
\definecolor{messagecolor}{rgb}{0, 0, 0}
\definecolor{warningcolor}{rgb}{1, 0, 1}
\definecolor{errorcolor}{rgb}{1, 0, 0}
\newenvironment{knitrout}{}{} % an empty environment to be redefined in TeX

\usepackage{alltt}
%\VignetteIndexEntry{RcppOctave}
%\VignetteDepends{RcppOctave,knitr,bibtex}
%\VignetteCompiler{knitr}
%\VignetteEngine{knitr::knitr}




\usepackage{fancyvrb}
\usepackage{indentfirst}
\usepackage{amsmath} %math symbols
\usepackage{amssymb} %extra math symbols
\usepackage{xspace}
\usepackage{url}
\usepackage[colorlinks]{hyperref}
\usepackage{a4wide}
\setlength{\oddsidemargin}{0pt}


% TABLE OF CONTENTS
\usepackage{tocloft}
\renewcommand{\cftsecleader}{\cftdotfill{\cftdotsep}}
\renewcommand{\cftsecfont}{\small}
\usepackage[toc]{multitoc}

\newcommand{\matlab}{Matlab$^\circledR$\xspace}

% add preamble from pkgmaker
%%%% PKGMAKER COMMANDS %%%%%%
\usepackage{xspace}

% R
\let\proglang=\textit
\let\code=\texttt 
\newcommand{\Rcode}{\code}
\newcommand{\pkgname}[1]{\textit{#1}\xspace}
\newcommand{\Rpkg}[1]{\pkgname{#1} package\xspace}
\newcommand{\citepkg}[1]{\cite{#1}}

% CRAN
\newcommand{\CRANurl}[1]{\url{http://cran.r-project.org/package=#1}}
%% CRANpkg
\makeatletter
\def\CRANpkg{\@ifstar\@CRANpkg\@@CRANpkg}
\def\@CRANpkg#1{\href{http://cran.r-project.org/package=#1}{\pkgname{#1}}\footnote{\CRANurl{#1}}}
\def\@@CRANpkg#1{\href{http://cran.r-project.org/package=#1}{\pkgname{#1}} package\footnote{\CRANurl{#1}}}
\makeatother
%% citeCRANpkg
\makeatletter
\def\citeCRANpkg{\@ifstar\@citeCRANpkg\@@citeCRANpkg}
\def\@citeCRANpkg#1{\CRANpkg{#1}\cite*{Rpackage:#1}}
\def\@@citeCRANpkg#1{\CRANpkg{#1}~\cite{Rpackage:#1}}
\makeatother
\newcommand{\CRANnmf}{\href{http://cran.r-project.org/package=NMF}{CRAN}}
\newcommand{\CRANnmfURL}{\url{http://cran.r-project.org/package=NMF}}

% Bioconductor
\newcommand{\BioCurl}[1]{\url{http://www.bioconductor.org/packages/release/bioc/html/#1.html}}
\newcommand{\BioCpkg}[1]{\href{http://www.bioconductor.org/packages/release/bioc/html/#1.html}{\pkgname{#1}} package\footnote{\BioCurl{#1}}}
\newcommand{\citeBioCpkg}[1]{\BioCpkg{#1}~\cite{Rpackage:#1}}
% Bioconductor annotation
\newcommand{\BioCAnnurl}[1]{\url{http://www.bioconductor.org/packages/release/data/annotation/html/#1.html}}
\newcommand{\BioCAnnpkg}[1]{\href{http://www.bioconductor.org/packages/release/data/annotation/html/#1.html}{\Rcode{#1}} annotation package\footnote{\BioCAnnurl{#1}}}
\newcommand{\citeBioCAnnpkg}[1]{\BioCAnnpkg{#1}~\cite{Rpackage:#1}}

% GEO
\newcommand{\GEOurl}[1]{\href{http://www.ncbi.nlm.nih.gov/geo/query/acc.cgi?acc=#1}{#1}\xspace}
\newcommand{\GEOhref}[1]{\GEOurl{#1}\footnote{\url{http://www.ncbi.nlm.nih.gov/geo/query/acc.cgi?acc=#1}}}

% ArrayExpress
\newcommand{\ArrayExpressurl}[1]{\href{http://www.ebi.ac.uk/arrayexpress/experiments/#1}{#1}\xspace}
\newcommand{\ArrayExpresshref}[1]{\ArrayExpressurl{#1}\footnote{\url{http://www.ebi.ac.uk/arrayexpress/experiments/#1}}}

%%%% END: PKGMAKER COMMANDS %%%%%%



% REFERENCES
\usepackage[minnames=1,maxnames=2]{biblatex}
\AtEveryCitekey{\clearfield{url}}
\bibliography{Rpackages}
\bibliography{/tmp/Rpkglib_6b477de88d3d/RcppOctave/REFERENCES}


\newcommand{\citet}[1]{\citeauthor{#1}~\cite{#1}}
%%

\newcommand{\R}{\proglang{R}\xspace}
\newcommand{\graphwidth}{0.9\columnwidth}
%\newcommand{\revision}[2][]{\textcolor{blue}{#2}{\bf \textcolor{blue}{$\blacksquare$ REV[#1]}}}

% Octave
\newcommand{\octave}{\proglang{Octave}\xspace}

% clever references
\usepackage[noabbrev, capitalise, nameinlink]{cleveref}

\author{Renaud Gaujoux}
\title{\pkgname{RcppOctave}: Seamless Interface to Octave -- and Matlab}
\date{\Rpkg{RcppOctave} -- Version 0.14
[March  5, 2014]\footnote{This vignette was built using \octave
3.8.0}}
\IfFileExists{upquote.sty}{\usepackage{upquote}}{}
\begin{document}
\maketitle

\begin{abstract}
The \Rpkg{RcppOctave} provides a direct interface to \octave from
\R.
It allows \\octave functions to be called from an \R session,
in a similar way \proglang{C/C++} or \proglang{Fortran} functions are called using the base function~\code{.Call}.
Since \octave uses a language that is mostly compatible with \matlab,
\pkgname{RcppOctave} may also be used to run Matlab m-files.
This package was originally developed to facilitate the port and comparison of R
and Matlab code.
In particular, it provides \octave modules that redefine
\octave default random number generator functions, so that they call
\proglang{R} own dedicated functions.
This enables to also reproduce and compare stochastic computations.
\end{abstract}

%These may also be used in any Octave session, independently of R, increasing
% the range of RNGs available in Octave.
\noindent\hrulefill
\tableofcontents
\noindent\hrulefill

\section{Introduction}

In many research fields, source code of algorithms and statistical methods are
published as Matlab files (the so called m-files).
While such code is generally released under public Open Source licenses like the
GNU Public Licenses (GPLs) \cite{gnuGPL}, effectively running or using it
require either to have \matlab, which is a nice but expensive proprietary
software\footnote{\url{http://www.mathworks.com}}, or to be/get -- at least -- a bit familiar with \octave \cite{Eaton2002}, which is free and open source, and is able to read and execute m-files, as long as they do not require Matlab-specific functions.
However, \proglang{R} users may have neither Matlab license, nor the
time/will to become \octave-skilled, and yet want to use algorithms written in
Matlab/\octave for their analyses and research.

Being able to run m-files or selectively use \octave functionalities
directly from \proglang{R} can greatly alleviate a process that otherwise
typically implies exporting/importing data between the two environments via
files on disk, as well as dealing with a variety of issues including
rounding errors, format compatibility or subtle implementation differences,
that all may lead to intricate hard-to-debug situations.
Even if one eventually wants to rewrite or optimise a given algorithm in plain
\proglang{R} or in \proglang{C/C++}, and therefore remove any dependency to
\octave, it is important to test the correctness of the port by
comparing its results with the original implementation.
Also, a direct interface allows users to stick to their preferred computing
environment, in which they are more comfortable and productive.

An \proglang{R} package called \pkgname{ROctave}
\footnote{\url{http://www.omegahat.org/ROctave}} does exist, and intends to
provide an interface between \proglang{R} and \octave, but appears
to be outdated (2002), and does not work out of the box with recent version of
\octave.
A more recent forum
post\footnote{\url{http://octave.1599824.n4.nabble.com/ROctave-bindings-for-2-1-73-2-9-x-td1602060.html}}
brought back some interest on binding these two environments, but
apparently without any following.

The \citeCRANpkg{RcppOctave} described in this vignette aims at filling the gap
and facilitating the usage of \proglang{Octave/Matlab} code from \proglang{R}, by providing a lean interface that enables direct and easy interaction with an embedded \octave session.
The package's name was chosen both to differentiate it from the
existing \Rpkg{ROctave}, and to reflect its use and integration of the
\proglang{C++} framework defined by the \citeCRANpkg{Rcpp}.

\section{Objectives \& Features}
The ultimate objective of \pkgname{RcppOctave} is to provide a two-way
interface between \proglang{R} and \octave, i.e. that allows calling
\octave from \proglang{R} and vice-versa.
The interface intends to be lean and as transparent as possible, as well
as providing convenient utilities to perform commonly needed tasks (e.g. source
files, browse documentation).

Currently, the package focuses on accessing \octave functionalities
from \proglang{R} with:

\begin{itemize}
 \item An out-of-the-box-working embedded \octave session;
 \item Ability to run/source m-files from \proglang{R};
 \item Ability to evaluate \octave statements and function calls from
 \proglang{R};
 \item Ability to call \proglang{R} functions in \octave
 code\footnote{Currently only when run from \proglang{R} through
 \pkgname{RcppOctave}.};
 \item Transparent passage of variables between \proglang{R} and \octave;
 \item Reproducibility of computations, including stochastic computations, in both environment;
\end{itemize}

Future development should provide similar reverse capabilities, i.e. an out
of the box embedded R session, typically via the \CRANpkg{RInside}.

\section{OS requirements}

The package has been developped and tested under Linux (Ubuntu), and has
notably been reported to work fine on other Linux distributions.
Developments to make it run on Windows and Mac recently started, and has
been so far relatively successful. 

\subsection{Linux}
The only requirement on Linux machines is to have Octave $\geq 3.2.4$
and its development files installed, although a more recent version ($\geq
3.6$) is recommended to get full functionnalities.

On Debian/Ubuntu this amounts to:

\begin{description}
  \item[Octave $\geq$ 3.6:] (works out of the box):
  
\begin{verbatim}
# install octave and development files
sudo apt-get install octave liboctave-dev
# install as usual in R
Rscript -e "install.packages('RcppOctave')"
\end{verbatim}

\item[Octave 3.2.4:] (might require extra command)
\begin{verbatim}
# install octave and development files
sudo apt-get install octave3.2 octave3.2-headers
# requires to explicitly export Octave lib directory 
export LD_LIBRARY_PATH=$LD_LIBRARY_PATH:`octave-config -p OCTLIBDIR`
# install as usual in R
Rscript -e "install.packages('RcppOctave')"
\end{verbatim}

\end{description}

\subsection{Windows}

Support for Windows started with version \emph{0.11}.
Developments and tests are performed on Windows 7 using the following
settings:

\begin{description}
  \item[Rtools:] the package contains C++ source files that need to be
  compiled, which means that \emph{Rtools} needs to be installed, with its
  \texttt{bin/} sub-directory in the system PATH.
  
  See \url{http://cran.r-project.org/bin/windows/Rtools/} for how to install the
  version of \emph{Rtools} compatible with your \emph{R} version;
  \item[Octave:] development was performed using the \emph{mingw} version of
  Octave, which can be installed as decribed in the Octave wiki:

\url{http://wiki.octave.org/Octave_for_Windows#Octave-3.6.4-mingw_.2B_octaveforge_pkgs}

  Octave binary \texttt{bin/} sub-directory (e.g.,
  \texttt{C:\textbackslash Octave\textbackslash Octave3.6.4\_gcc4.6.2\textbackslash bin}) must be in the system PATH as well, preferably after
  \emph{Rtools} own \texttt{bin/} sub-directory.
\end{description}

\subsection{Mac OS}
Support for Mac OS is not yet official, but is currently being investigating.
Preliminary discussion(s) on how to install and run under Mac can be found here:

\url{http://lists.r-forge.r-project.org/pipermail/rcppoctave-user/2013-October/000024.html}

The installation procedure under investigation is based on the Octave version
provided by \emph{homebrew}\footnote{\url{http://brew.sh/}}:

\begin{enumerate}
  \item install XCode and its Command Line Tools
  \item install homebrew
  \item add the homebrew/science repository (tap in brewing language):

\begin{verbatim}
brew tap homebrew/science
brew update && brew upgrade
brew tap --repair #may not be necessary
brew install gfortran
brew install octave
\end{verbatim}
\end{enumerate}

\section{Accessing Octave from R}

The \Rpkg{RcppOctave} defines the function \code{.CallOctave}, which acts as a
single entry point for calling \octave functions from \proglang{R}.
In order to make common function calls easier (e.g. \code{eval}), other utility
functions are defined, which essentially wraps a call to \code{.CallOctave}, but
enhance argument handling and result formating.

\subsection{Core interface: \texttt{.CallOctave}}
The function \code{.CallOctave} calls an \octave \underline{function}
from R, mimicking the way native \proglang{C/C++} functions are called with
\code{.Call}.

\subsubsection{Overview}

The function \code{.CallOctave} takes the name of an \octave function (in its
first argument \code{.NAME}) and pass the remaining arguments directly to the
\octave function -- except for the two special arguments \code{argout} (see next
section) and \code{unlist}.
Note that \octave function arguments are not named and positional, meaning that
they must be passed in the correct order.
Input names are simply ignored by \code{.CallOctave}.
Calling any \octave function is then as simple as: 

\begin{knitrout}
\definecolor{shadecolor}{rgb}{0.969, 0.969, 0.969}\color{fgcolor}\begin{kframe}
\begin{alltt}
\hlkwd{.CallOctave}\hlstd{(}\hlstr{"version"}\hlstd{)}
\end{alltt}
\begin{verbatim}
## [1] "3.8.0"
\end{verbatim}
\begin{alltt}
\hlkwd{.CallOctave}\hlstd{(}\hlstr{"sqrt"}\hlstd{,} \hlnum{10}\hlstd{)}
\end{alltt}
\begin{verbatim}
## [1] 3.162
\end{verbatim}
\begin{alltt}
\hlkwd{.CallOctave}\hlstd{(}\hlstr{"eye"}\hlstd{,} \hlnum{3}\hlstd{)}
\end{alltt}
\begin{verbatim}
##      [,1] [,2] [,3]
## [1,]    1    0    0
## [2,]    0    1    0
## [3,]    0    0    1
\end{verbatim}
\begin{alltt}
\hlkwd{.CallOctave}\hlstd{(}\hlstr{"eye"}\hlstd{,} \hlnum{3}\hlstd{,} \hlnum{2}\hlstd{)}
\end{alltt}
\begin{verbatim}
##      [,1] [,2]
## [1,]    1    0
## [2,]    0    1
## [3,]    0    0
\end{verbatim}
\end{kframe}
\end{knitrout}


\subsubsection{Controlling output values}
\label{sec:argout}

\octave functions have the interesting feature of being able to compute
and return a variable number of output values, depending on the number of output
variables specified in the statement.
Hence, a call to an \octave function requires passing both its parameters and
the number of desired output values.

The following sample code illustrates this concept using the function
\code{svd}\footnote{This sample code is extracted from the manpage for
\code{svd}. See \code{o\_help(svd)} for more details.}:

\begin{Verbatim}[frame=single]
% single output variable: eigen values only
S = svd(A);

% 3 output variables: complete SVD decomposition  
[U, S, V] = svd(A);
\end{Verbatim}

The default behaviour of \code{.CallOctave} is to try to
detect the maximum number of output variables, as well as their names, and
return them all.
This should be suitable for most common cases, especially for functions
defined by the user in plain m-files, but does not work for functions defined in
compiled modules (see examples with in the next section).
Hence the default is to return the maximum number of output values if it can be
detected, or only the first one.

For some functions, however, this behaviour may not be ideal, and complete
control on the return values is possible via the special argument \code{argout}.
The next section illustrates different situations and use case scenarios.

\subsubsection{Examples}

A sample m-file (i.e. a function definition file) is shipped with any 
\pkgname{RcppOctave} installation in the ``scripts/" sub-directory and provides
some examples of different types of \octave functions:

\begin{Verbatim}[frame=single]
%%%%%%%%%%%%%%%%%%%%%%%%%%%%%%%%%%%%%%%%%%%%%%
% Example file for the R package RcppOctave
%%%%%%%%%%%%%%%%%%%%%%%%%%%%%%%%%%%%%%%%%%%%%%

function [a] = fun1()
	a = rand(1,4);
end

function [a,b,c] = fun2()
	a = rand(1,4);
	b = rand(2,3);
	c = "some text";
end

function fun_noargout(x) 
	% no effect outside the function
	y = 1;
	printf("%% Printed from Octave: x="), disp(x);
end

function [s] = fun_varargin(varargin)
  if (nargin==0)
	s = 0;
  else
	s = varargin{1} + varargin{2} + varargin{3};
  endif
end

function [u, s, v] = fun_varargout()

	if (nargout == 1) u = 1; 
	elseif (nargout == 3)
		u = 10; s = 20; v = 30; 
	else usage("Expecting 1 or 3 output variables.");
	endif; 
end



\end{Verbatim}

These definitions can be loaded in the \octave session via the function
\code{sourceExamples}.

\begin{knitrout}
\definecolor{shadecolor}{rgb}{0.969, 0.969, 0.969}\color{fgcolor}\begin{kframe}
\begin{alltt}
\hlcom{# source example function definitions from RcppOctave installation}
\hlkwd{sourceExamples}\hlstd{(}\hlstr{"ex_functions.m"}\hlstd{)}
\hlcom{# several functions are now defined}
\hlkwd{o_ls}\hlstd{()}
\end{alltt}
\begin{verbatim}
## [1] "fun1"          "fun2"          "fun_noargout"  "fun_varargin" 
## [5] "fun_varargout"
\end{verbatim}
\end{kframe}
\end{knitrout}


The functions \code{fun1}, \code{fun2}, \code{fun\_noargout}, and
\code{fun\_varargin} perform the same computations independently of the number
of output.
For these a default call to \code{.CallOctave} is enough to get their full
functionalities:

\begin{knitrout}
\definecolor{shadecolor}{rgb}{0.969, 0.969, 0.969}\color{fgcolor}\begin{kframe}
\begin{alltt}
\hlcom{# single output value}
\hlkwd{.CallOctave}\hlstd{(}\hlstr{"fun1"}\hlstd{)}
\end{alltt}
\begin{verbatim}
## [1] 0.5170 0.6824 0.3656 0.4314
\end{verbatim}
\begin{alltt}
\hlcom{# 3 output values}
\hlkwd{.CallOctave}\hlstd{(}\hlstr{"fun2"}\hlstd{)}
\end{alltt}
\begin{verbatim}
## $a
## [1] 0.0503 0.4100 0.1331 0.6679
## 
## $b
##        [,1]   [,2]   [,3]
## [1,] 0.2915 0.1338 0.6036
## [2,] 0.4920 0.8285 0.7057
## 
## $c
## [1] "some text"
\end{verbatim}
\begin{alltt}
\hlcom{# no output value}
\hlkwd{.CallOctave}\hlstd{(}\hlstr{"fun_noargout"}\hlstd{,} \hlnum{1}\hlstd{)}
\end{alltt}
\begin{verbatim}
## % Printed from Octave: x= 1
\end{verbatim}
\begin{alltt}
\hlkwd{.CallOctave}\hlstd{(}\hlstr{"fun_noargout"}\hlstd{,} \hlstr{"abc"}\hlstd{)}
\end{alltt}
\begin{verbatim}
## % Printed from Octave: x=abc
\end{verbatim}
\begin{alltt}
\hlcom{# variable number of arguments}
\hlkwd{.CallOctave}\hlstd{(}\hlstr{"fun_varargin"}\hlstd{)}
\end{alltt}
\begin{verbatim}
## [1] 0
\end{verbatim}
\begin{alltt}
\hlkwd{.CallOctave}\hlstd{(}\hlstr{"fun_varargin"}\hlstd{,} \hlnum{1}\hlstd{,} \hlnum{2}\hlstd{,} \hlnum{3}\hlstd{)}
\end{alltt}
\begin{verbatim}
## [1] 6
\end{verbatim}
\end{kframe}
\end{knitrout}


The function \code{fun\_varargout} however, behaves differently when called
with 1, 2 or 3 output variables, performing different computations.
Since it is defined in a m-file, the maximum set of output variables is
detectable and the default behaviour is then to call it asking for 3 output
variables.
The other types of computations can be obtained using argument \code{argout}:

\begin{knitrout}
\definecolor{shadecolor}{rgb}{0.969, 0.969, 0.969}\color{fgcolor}\begin{kframe}
\begin{alltt}
\hlkwd{.CallOctave}\hlstd{(}\hlstr{"fun_varargout"}\hlstd{)}
\end{alltt}
\begin{verbatim}
## $u
## [1] 10
## 
## $s
## [1] 20
## 
## $v
## [1] 30
\end{verbatim}
\begin{alltt}
\hlkwd{.CallOctave}\hlstd{(}\hlstr{"fun_varargout"}\hlstd{,} \hlkwc{argout} \hlstd{=} \hlnum{1}\hlstd{)}
\end{alltt}
\begin{verbatim}
## [1] 1
\end{verbatim}
\begin{alltt}
\hlcom{# this should throw an error}
\hlkwd{try}\hlstd{(}\hlkwd{.CallOctave}\hlstd{(}\hlstr{"fun_varargout"}\hlstd{,} \hlkwc{argout} \hlstd{=} \hlnum{2}\hlstd{))}
\end{alltt}


{\ttfamily\noindent\bfseries\color{errorcolor}{\#\# Error: RcppOctave - error in Octave function `fun\_varargout`:\\\#\#\ \  error:\ \  fun\_varargout at line 34, column 7}}\end{kframe}
\end{knitrout}


Argument \code{argout} may also be used to specify names for the output values.
This is useful for functions defined in compiled modules (e.g. \code{svd}) for
which expected outputs are not detectable (output names in particular), or when
limiting the number of output variables in functions defined in m-files.
Indeed, in this latter case, it is not safe to infer the names based on those
defined for the complete output, as these may not be relevant anymore:

\begin{knitrout}
\definecolor{shadecolor}{rgb}{0.969, 0.969, 0.969}\color{fgcolor}\begin{kframe}
\begin{alltt}
\hlcom{# single output variable: result is S}
\hlkwd{.CallOctave}\hlstd{(}\hlstr{"svd"}\hlstd{,} \hlkwd{matrix}\hlstd{(}\hlnum{1}\hlopt{:}\hlnum{4}\hlstd{,} \hlnum{2}\hlstd{))}
\end{alltt}
\begin{verbatim}
##       [,1]
## [1,] 5.465
## [2,] 0.366
\end{verbatim}
\begin{alltt}
\hlcom{# 3 output variables: results is [U,S,V]}
\hlkwd{.CallOctave}\hlstd{(}\hlstr{"svd"}\hlstd{,} \hlkwd{matrix}\hlstd{(}\hlnum{1}\hlopt{:}\hlnum{4}\hlstd{,} \hlnum{2}\hlstd{),} \hlkwc{argout} \hlstd{=} \hlnum{3}\hlstd{)}
\end{alltt}
\begin{verbatim}
## [[1]]
##         [,1]    [,2]
## [1,] -0.5760 -0.8174
## [2,] -0.8174  0.5760
## 
## [[2]]
##       [,1]  [,2]
## [1,] 5.465 0.000
## [2,] 0.000 0.366
## 
## [[3]]
##         [,1]    [,2]
## [1,] -0.4046  0.9145
## [2,] -0.9145 -0.4046
\end{verbatim}
\begin{alltt}
\hlcom{# specify output names (and therefore number of output variables)}
\hlkwd{.CallOctave}\hlstd{(}\hlstr{"svd"}\hlstd{,} \hlkwd{matrix}\hlstd{(}\hlnum{1}\hlopt{:}\hlnum{4}\hlstd{,} \hlnum{2}\hlstd{),} \hlkwc{argout} \hlstd{=} \hlkwd{c}\hlstd{(}\hlstr{"U"}\hlstd{,} \hlstr{"S"}\hlstd{,} \hlstr{"V"}\hlstd{))}
\end{alltt}
\begin{verbatim}
## $U
##         [,1]    [,2]
## [1,] -0.5760 -0.8174
## [2,] -0.8174  0.5760
## 
## $S
##       [,1]  [,2]
## [1,] 5.465 0.000
## [2,] 0.000 0.366
## 
## $V
##         [,1]    [,2]
## [1,] -0.4046  0.9145
## [2,] -0.9145 -0.4046
\end{verbatim}
\end{kframe}
\end{knitrout}


Note that it is quite possible for a compiled function to only accept
calls with at least 2 output variables.
In such cases, \code{.CallOctave} calls must always specify argument
\code{argout}.

\subsection{Direct interface: the \texttt{.O} object}

An alternative and convenient shortcut interface is defined by the S4-class
\code{Octave}.
At load time, an instance of this class, an object named \code{.O}, is
initialised and exported from \pkgname{RcppOctave}'s namespace.
Using the \code{.O} object, calls to \octave functions are more compact: 

\begin{knitrout}
\definecolor{shadecolor}{rgb}{0.969, 0.969, 0.969}\color{fgcolor}\begin{kframe}
\begin{alltt}
\hlstd{.O}
\end{alltt}
\begin{verbatim}
##  <Octave Interface>
##  - Use `$x` to call Octave function or get variable x.
##  - Use `$x <- val` to assign a value val to the Octave variable x.
\end{verbatim}
\begin{alltt}
\hlstd{.O}\hlopt{$}\hlkwd{version}\hlstd{()}
\end{alltt}
\begin{verbatim}
## [1] "3.8.0"
\end{verbatim}
\begin{alltt}
\hlstd{.O}\hlopt{$}\hlkwd{eye}\hlstd{(}\hlnum{3}\hlstd{)}
\end{alltt}
\begin{verbatim}
##      [,1] [,2] [,3]
## [1,]    1    0    0
## [2,]    0    1    0
## [3,]    0    0    1
\end{verbatim}
\begin{alltt}
\hlstd{.O}\hlopt{$}\hlkwd{svd}\hlstd{(}\hlkwd{matrix}\hlstd{(}\hlnum{1}\hlopt{:}\hlnum{4}\hlstd{,} \hlnum{2}\hlstd{))}
\end{alltt}
\begin{verbatim}
##       [,1]
## [1,] 5.465
## [2,] 0.366
\end{verbatim}
\begin{alltt}
\hlcom{# argout can still be specified}
\hlstd{.O}\hlopt{$}\hlkwd{svd}\hlstd{(}\hlkwd{matrix}\hlstd{(}\hlnum{1}\hlopt{:}\hlnum{4}\hlstd{,} \hlnum{2}\hlstd{),} \hlkwc{argout} \hlstd{=} \hlnum{3}\hlstd{)}
\end{alltt}
\begin{verbatim}
## [[1]]
##         [,1]    [,2]
## [1,] -0.5760 -0.8174
## [2,] -0.8174  0.5760
## 
## [[2]]
##       [,1]  [,2]
## [1,] 5.465 0.000
## [2,] 0.000 0.366
## 
## [[3]]
##         [,1]    [,2]
## [1,] -0.4046  0.9145
## [2,] -0.9145 -0.4046
\end{verbatim}
\end{kframe}
\end{knitrout}


\subsubsection{Manipulating variables}
The \code{.O} object facilitates manipulating single \octave
variables, as it emulates an \proglang{R} environment-like object whose elements
would be the objects available in the current \octave
embedded session:

\begin{knitrout}
\definecolor{shadecolor}{rgb}{0.969, 0.969, 0.969}\color{fgcolor}\begin{kframe}
\begin{alltt}
\hlcom{# define a variable}
\hlstd{.O}\hlopt{$}\hlstd{myvar} \hlkwb{<-} \hlnum{1}\hlopt{:}\hlnum{5}
\hlcom{# retrieve value}
\hlstd{.O}\hlopt{$}\hlstd{myvar}
\end{alltt}
\begin{verbatim}
## [1] 1 2 3 4 5
\end{verbatim}
\begin{alltt}
\hlcom{# assign and retrieve new value}
\hlstd{.O}\hlopt{$}\hlstd{myvar} \hlkwb{<-} \hlnum{10}
\hlstd{.O}\hlopt{$}\hlstd{myvar}
\end{alltt}
\begin{verbatim}
## [1] 10
\end{verbatim}
\begin{alltt}
\hlcom{# remove}
\hlstd{.O}\hlopt{$}\hlstd{myvar} \hlkwb{<-} \hlkwa{NULL}
\hlcom{# this should now throw an error since 'myvar' does not exist anymore}
\hlkwd{try}\hlstd{(.O}\hlopt{$}\hlstd{myvar)}
\end{alltt}


{\ttfamily\noindent\bfseries\color{errorcolor}{\#\# Error: RcppOctave::o\_get - Could not find an Octave object named 'myvar'.}}\end{kframe}
\end{knitrout}


\subsubsection{Calling functions}
As illustrated above, \octave functions can be called through the
\code{.O} object, by passing specifying its arguments as a function call:
\begin{knitrout}
\definecolor{shadecolor}{rgb}{0.969, 0.969, 0.969}\color{fgcolor}\begin{kframe}
\begin{alltt}
\hlcom{# density of x=5 for Poisson(2)}
\hlstd{.O}\hlopt{$}\hlkwd{poisspdf}\hlstd{(}\hlnum{5}\hlstd{,} \hlnum{2}\hlstd{)}
\end{alltt}
\begin{verbatim}
## [1] 0.03609
\end{verbatim}
\begin{alltt}
\hlcom{# E.g. compare with R own function}
\hlkwd{dpois}\hlstd{(}\hlnum{5}\hlstd{,} \hlnum{2}\hlstd{)}
\end{alltt}
\begin{verbatim}
## [1] 0.03609
\end{verbatim}
\end{kframe}
\end{knitrout}


They may also be retrieved as \proglang{R} functions in a similar way as
variables, and called in subsequent statements:

\begin{knitrout}
\definecolor{shadecolor}{rgb}{0.969, 0.969, 0.969}\color{fgcolor}\begin{kframe}
\begin{alltt}
\hlcom{# retrieve Octave function}
\hlstd{f} \hlkwb{<-} \hlstd{.O}\hlopt{$}\hlstd{poisspdf}
\hlstd{f}
\end{alltt}
\begin{verbatim}
## <OctaveFunction::`poisspdf`>
\end{verbatim}
\begin{alltt}
\hlcom{# call (in Octave)}
\hlkwd{f}\hlstd{(}\hlnum{5}\hlstd{,} \hlnum{2}\hlstd{)}
\end{alltt}
\begin{verbatim}
## [1] 0.03609
\end{verbatim}
\end{kframe}
\end{knitrout}



\subsubsection{Auto-completion}
An advantage of using the \code{.O} object is that it has auto-completion
capabilities similar to the \proglang{R} console.
This greatly helps and speeds up the interaction with the current embedded
\octave session.
For example, typing \code{.O\$std} + \code{TAB} + \code{TAB} will show all
functions or variables available in the current session, that start with ``std". 

\subsection{Utility functions}

The \Rpkg{RcppOctave} defines some utilities to enhance the interaction with
\octave, and alleviate calls to a set of commonly used
\octave functions.
All these functions start with the prefix ``o\_" (e.g.
\code{o\_source}), so that they can be listed by typing \code{o\_ + TAB + TAB}
in the \proglang{R} console.
Their names have been chosen to reflect the corresponding \octave
function, and, in some cases, aliases matching standard \proglang{R} names are
also provided, so that users not familiar with \octave can find their
way quickly (e.g. \code{o\_rm} is an alias to \code{o\_clear}).

\subsubsection{Assign/get variables}

The functions \code{o\_assign} and \code{o\_get} facilitates assigning variables
and retrieving objects (variables or functions).
Variables may be assigned or retrieved individually in separate calls to
\code{o\_assign} or \code{o\_get}\footnote{This would be similar to using the
\code{.O} object as described above}, or simultaneously in a variety of ways
(see \code{?o\_get} for more details and examples):

\begin{knitrout}
\definecolor{shadecolor}{rgb}{0.969, 0.969, 0.969}\color{fgcolor}\begin{kframe}
\begin{alltt}
\hlcom{## ASSIGN}
\hlkwd{o_assign}\hlstd{(}\hlkwc{a} \hlstd{=} \hlnum{1}\hlstd{)}
\hlkwd{o_assign}\hlstd{(}\hlkwc{a} \hlstd{=} \hlnum{10}\hlstd{,} \hlkwc{b} \hlstd{=} \hlnum{20}\hlstd{)}
\hlkwd{o_assign}\hlstd{(}\hlkwd{list}\hlstd{(}\hlkwc{a} \hlstd{=} \hlnum{5}\hlstd{,} \hlkwc{b} \hlstd{=} \hlnum{6}\hlstd{,} \hlkwc{aaa} \hlstd{=} \hlnum{7}\hlstd{,} \hlkwc{aab} \hlstd{=} \hlkwd{list}\hlstd{(}\hlnum{1}\hlstd{,} \hlnum{2}\hlstd{,} \hlnum{3}\hlstd{)))}

\hlcom{## GET get all variables}
\hlkwd{str}\hlstd{(}\hlkwd{o_get}\hlstd{())}
\end{alltt}
\begin{verbatim}
## List of 4
##  $ a  : num 5
##  $ aaa: num 7
##  $ aab:List of 3
##   ..$ : num 1
##   ..$ : num 2
##   ..$ : num 3
##  $ b  : num 6
\end{verbatim}
\begin{alltt}
\hlcom{# selected variables}
\hlkwd{o_get}\hlstd{(}\hlstr{"a"}\hlstd{)}
\end{alltt}
\begin{verbatim}
## [1] 5
\end{verbatim}
\begin{alltt}
\hlkwd{o_get}\hlstd{(}\hlstr{"a"}\hlstd{,} \hlstr{"b"}\hlstd{)}
\end{alltt}
\begin{verbatim}
## $a
## [1] 5
## 
## $b
## [1] 6
\end{verbatim}
\begin{alltt}
\hlcom{# rename on the fly}
\hlkwd{o_get}\hlstd{(}\hlkwc{c} \hlstd{=} \hlstr{"a"}\hlstd{,} \hlkwc{d} \hlstd{=} \hlstr{"b"}\hlstd{)}
\end{alltt}
\begin{verbatim}
## $c
## [1] 5
## 
## $d
## [1] 6
\end{verbatim}
\begin{alltt}
\hlcom{# o_get throw an error for objects that do not exist}
\hlkwd{try}\hlstd{(}\hlkwd{o_get}\hlstd{(}\hlstr{"xxxxx"}\hlstd{))}
\end{alltt}


{\ttfamily\noindent\bfseries\color{errorcolor}{\#\# Error: RcppOctave::o\_get - Could not find an Octave object named 'xxxxx'.}}\begin{alltt}
\hlcom{# but suggests potential matches}
\hlkwd{try}\hlstd{(}\hlkwd{o_get}\hlstd{(}\hlstr{"aa"}\hlstd{))}
\end{alltt}


{\ttfamily\noindent\bfseries\color{errorcolor}{\#\# Error: RcppOctave::o\_get - Could not find an Octave object named 'aa'.\\\#\#\ \ \ \ \ \ \ \ Match(es): aaa aab}}\begin{alltt}
\hlcom{# get a function}
\hlstd{f} \hlkwb{<-} \hlkwd{o_get}\hlstd{(}\hlstr{"svd"}\hlstd{)}
\hlstd{f}
\end{alltt}
\begin{verbatim}
## <OctaveFunction::`svd`>
\end{verbatim}
\end{kframe}
\end{knitrout}


\subsubsection{Evaluate single statements}

To evaluate a single statement, one can use the \code{o\_eval} function, that
can also evaluate a list of statements sequentially:

\begin{knitrout}
\definecolor{shadecolor}{rgb}{0.969, 0.969, 0.969}\color{fgcolor}\begin{kframe}
\begin{alltt}
\hlcom{# assign variable 'a'}
\hlkwd{o_eval}\hlstd{(}\hlstr{"a=1"}\hlstd{)}
\end{alltt}
\begin{verbatim}
## [1] 1
\end{verbatim}
\begin{alltt}
\hlkwd{o_eval}\hlstd{(}\hlstr{"a"}\hlstd{)}  \hlcom{# or .O$a}
\end{alltt}
\begin{verbatim}
## [1] 1
\end{verbatim}
\begin{alltt}
\hlkwd{o_eval}\hlstd{(}\hlstr{"a=svd(rand(3))"}\hlstd{)}
\end{alltt}
\begin{verbatim}
##         [,1]
## [1,] 1.51788
## [2,] 0.25908
## [3,] 0.01084
\end{verbatim}
\begin{alltt}
\hlstd{.O}\hlopt{$}\hlstd{a}
\end{alltt}
\begin{verbatim}
##         [,1]
## [1,] 1.51788
## [2,] 0.25908
## [3,] 0.01084
\end{verbatim}
\begin{alltt}
\hlcom{# eval a list of statements}
\hlstd{l} \hlkwb{<-} \hlkwd{o_eval}\hlstd{(}\hlstr{"a=rand(1, 2)"}\hlstd{,} \hlstr{"b=randn(1, 2)"}\hlstd{,} \hlstr{"rand(1, 3)"}\hlstd{)}
\hlstd{l}
\end{alltt}
\begin{verbatim}
## [[1]]
## [1] 0.2627 0.3801
## 
## [[2]]
## [1] -1.55072 -0.06703
## 
## [[3]]
## [1] 0.3170 0.1008 0.9161
\end{verbatim}
\begin{alltt}
\hlcom{# variables 'a' and 'b' were assigned the new values}
\hlkwd{identical}\hlstd{(}\hlkwd{list}\hlstd{(.O}\hlopt{$}\hlstd{a, .O}\hlopt{$}\hlstd{b), l[}\hlnum{1}\hlopt{:}\hlnum{2}\hlstd{])}
\end{alltt}
\begin{verbatim}
## [1] TRUE
\end{verbatim}
\begin{alltt}
\hlcom{# multiple statements are not supported by o_eval}
\hlkwd{try}\hlstd{(}\hlkwd{o_eval}\hlstd{(}\hlstr{"a=1; b=2"}\hlstd{))}
\end{alltt}


{\ttfamily\noindent\bfseries\color{errorcolor}{\#\# Error: RcppOctave - error in Octave function `eval`:\\\#\#\ \  error: eval: invalid use of statement list}}\begin{alltt}
\hlstd{.O}\hlopt{$}\hlstd{a}
\end{alltt}
\begin{verbatim}
## [1] 0.2627 0.3801
\end{verbatim}
\begin{alltt}
\hlcom{# argument CATCH allows for recovering from errors in statement}
\hlkwd{o_eval}\hlstd{(}\hlstr{"a=usage('ERROR: stop here')"}\hlstd{,} \hlkwc{CATCH} \hlstd{=} \hlstr{"c=3"}\hlstd{)}
\end{alltt}
\begin{verbatim}
## [1] 3
\end{verbatim}
\begin{alltt}
\hlstd{.O}\hlopt{$}\hlstd{a}
\end{alltt}
\begin{verbatim}
## [1] 0.2627 0.3801
\end{verbatim}
\begin{alltt}
\hlstd{.O}\hlopt{$}\hlstd{c}
\end{alltt}
\begin{verbatim}
## [1] 3
\end{verbatim}
\end{kframe}
\end{knitrout}


More details and examples are provided in the manual page \code{?o\_eval}.
If more than one statement is to be evaluated, then one should use the function
\code{o\_source}, with argument \code{text} as described in \cref{sec:o_source} below.

\subsubsection{Source m-files}
\label{sec:o_source}

\proglang{Octave/Matlab} code generally are generally provided as so
called m-files, which are plain text files that contain function definitions
and/or sequences of multiple commands that perform a given task.
This is the form most public third party algorithms are published.

The function \code{o\_source} allows to load these files in the current
\octave session, so that the object they define are available, or the
commands they contain are executed.
\pkgname{RcppOctave} ships an example m-file in the ``scripts/" sub-directory
of its installation:

\begin{knitrout}
\definecolor{shadecolor}{rgb}{0.969, 0.969, 0.969}\color{fgcolor}\begin{kframe}
\begin{alltt}
\hlcom{# clear all session}
\hlkwd{o_clear}\hlstd{(}\hlkwc{all} \hlstd{=} \hlnum{TRUE}\hlstd{)}
\hlkwd{o_ls}\hlstd{()}
\end{alltt}
\begin{verbatim}
## character(0)
\end{verbatim}
\begin{alltt}
\hlcom{# source example file from RcppOctave installation}
\hlstd{mfile} \hlkwb{<-} \hlkwd{system.file}\hlstd{(}\hlstr{"scripts/ex_source.m"}\hlstd{,} \hlkwc{package} \hlstd{=} \hlstr{"RcppOctave"}\hlstd{)}
\hlkwd{cat}\hlstd{(}\hlkwd{readLines}\hlstd{(mfile),} \hlkwc{sep} \hlstd{=} \hlstr{"\textbackslash{}n"}\hlstd{)}
\end{alltt}
\begin{verbatim}
## % Example m-file to illustrate the usage of the function o_source
## %
## % This file defines 3 dummy variables ('a','b' and 'c') 
## % and a dummy function 'abc', that adds up its three arguments.
## % 
## 
## a = 1;
## b = 2;
## c = 3;
## 
## function [res] = abc(x, y, z)
## 	res = x + y + z; 
## end
\end{verbatim}
\begin{alltt}
\hlkwd{o_source}\hlstd{(mfile)}
\hlcom{# Now objects 'a', 'b', and 'c' as well as the function 'abc' should be}
\hlcom{# defined:}
\hlkwd{o_ls}\hlstd{(}\hlkwc{long} \hlstd{=} \hlnum{TRUE}\hlstd{)}
\end{alltt}
\begin{verbatim}
##  <Octave session: 4 object(s)>
##  name size bytes    class global sparse complex nesting persistent
##     a  1x1     8   double  FALSE  FALSE   FALSE       1      FALSE
##     b  1x1     8   double  FALSE  FALSE   FALSE       1      FALSE
##     c  1x1     8   double  FALSE  FALSE   FALSE       1      FALSE
##   abc   NA    NA function   TRUE     NA      NA       1         NA
\end{verbatim}
\begin{alltt}
\hlcom{# }
\hlkwd{o_eval}\hlstd{(}\hlstr{"abc(2, 4, 6)"}\hlstd{)}
\end{alltt}
\begin{verbatim}
## [1] 12
\end{verbatim}
\begin{alltt}
\hlkwd{o_eval}\hlstd{(}\hlstr{"abc(a, b, c)"}\hlstd{)}
\end{alltt}
\begin{verbatim}
## [1] 6
\end{verbatim}
\end{kframe}
\end{knitrout}


This function can also conveniently be used to evaluate multiple statements
directly passed from the \R console as character strings via its argument
\code{text}:

\begin{knitrout}
\definecolor{shadecolor}{rgb}{0.969, 0.969, 0.969}\color{fgcolor}\begin{kframe}
\begin{alltt}
\hlkwd{o_source}\hlstd{(}\hlkwc{text} \hlstd{=} \hlstr{"clear a b c; a=100; a*sin(123)"}\hlstd{)}
\hlcom{# last statement is stored in automatic variable 'ans'}
\hlkwd{o_get}\hlstd{(}\hlstr{"a"}\hlstd{,} \hlstr{"ans"}\hlstd{)}
\end{alltt}
\begin{verbatim}
## $a
## [1] 100
## 
## $ans
## [1] -45.99
\end{verbatim}
\end{kframe}
\end{knitrout}


\subsubsection{List objects}

The function \code{o\_ls} (as used above) lists the objects (variables and
functions) that are defined in the current \octave embedded session.
It is an enhanced version over \octave standard listing functions such as
\code{who} (see \code{?o\_who}), which only lists variables, and not
user-defined functions.
With argument \code{long} it returns details about each variable and function,
in a similar way \code{whos} does (see \code{?o\_who}).

\begin{knitrout}
\definecolor{shadecolor}{rgb}{0.969, 0.969, 0.969}\color{fgcolor}\begin{kframe}
\begin{alltt}
\hlkwd{o_ls}\hlstd{()}
\end{alltt}
\begin{verbatim}
## [1] "a"   "abc"
\end{verbatim}
\begin{alltt}
\hlkwd{o_ls}\hlstd{(}\hlkwc{long} \hlstd{=} \hlnum{TRUE}\hlstd{)}
\end{alltt}
\begin{verbatim}
##  <Octave session: 2 object(s)>
##  name size bytes    class global sparse complex nesting persistent
##     a  1x1     8   double  FALSE  FALSE   FALSE       1      FALSE
##   abc   NA    NA function   TRUE     NA      NA       1         NA
\end{verbatim}
\begin{alltt}
\hlcom{# clear all (variables + functions)}
\hlkwd{o_clear}\hlstd{(}\hlkwc{all} \hlstd{=} \hlnum{TRUE}\hlstd{)}
\hlkwd{o_ls}\hlstd{()}
\end{alltt}
\begin{verbatim}
## character(0)
\end{verbatim}
\end{kframe}
\end{knitrout}


See \code{?o\_ls} for more details as well as \cref{sec:issues} for a
known issue in \octave versions older than 3.6.1.

\subsubsection{Browse documentation}

\octave has offers two ways of browsing documentation, via the functions
\code{help} and \code{doc}, which display a manual page for a given function and
lookup the whole documentation for a given topic respectively.

The \Rpkg{RcppOctave} provides wrapper for these two functions to enable
browsing \octave help pages in the way \R users are used to.
Hence, to access the manpage for a given function one types for example the
following, which displays using the \R function \code{file.show}:
\begin{knitrout}
\definecolor{shadecolor}{rgb}{0.969, 0.969, 0.969}\color{fgcolor}\begin{kframe}
\begin{alltt}
\hlkwd{o_help}\hlstd{(std)}
\end{alltt}
\end{kframe}
\end{knitrout}


To display all documentation about a topic one types for example the following,
opens the documentation using the GNU Info browser\footnote{At least on
Linux machines.}:
\begin{knitrout}
\definecolor{shadecolor}{rgb}{0.969, 0.969, 0.969}\color{fgcolor}\begin{kframe}
\begin{alltt}
\hlkwd{o_doc}\hlstd{(poisson)}
\end{alltt}
\end{kframe}
\end{knitrout}

Once the GNU Info browser is running, help for using it is available using the
command `Ctrl + h' -- as stated in the \octave documentation for \code{doc} (see
\code{o\_help(doc)}).

\subsection{Errors and warning handling}

All i/o messages written by Octave are redirected to R own i/o functions, with
errors and warnings generating corresponding messages in R\footnote{On
Windows, output redirection does not working properly and all output is
"mysteriously`` directly displayed by Octave}:
\begin{knitrout}
\definecolor{shadecolor}{rgb}{0.969, 0.969, 0.969}\color{fgcolor}\begin{kframe}
\begin{alltt}
\hlcom{# error}
\hlstd{res} \hlkwb{<-} \hlkwd{try}\hlstd{(}\hlkwd{.CallOctave}\hlstd{(}\hlstr{"error"}\hlstd{,} \hlstr{"this is an error in Octave"}\hlstd{))}
\end{alltt}


{\ttfamily\noindent\bfseries\color{errorcolor}{\#\# Error: RcppOctave - error in Octave function `error`:\\\#\#\ \  error: this is an error in Octave}}\begin{alltt}
\hlkwd{geterrmessage}\hlstd{()}
\end{alltt}
\begin{verbatim}
## [1] "Error in .CallOctave(\"error\", \"this is an error in Octave\") : \n  RcppOctave - error in Octave function `error`:\n  error: this is an error in Octave\n\n"
\end{verbatim}
\begin{alltt}
\hlcom{# warning}
\hlstd{res} \hlkwb{<-} \hlkwd{.CallOctave}\hlstd{(}\hlstr{"warning"}\hlstd{,} \hlstr{"this is a warning in Octave"}\hlstd{)}
\end{alltt}


{\ttfamily\noindent\color{warningcolor}{\#\# Warning: warning: this is a warning in Octave}}\end{kframe}
\end{knitrout}



\subsection{Low-level \textit{C/C++} interface}

\pkgname{RcppOctave} builds upon the \Rpkg{Rcpp}, and defines specialisation
for the \proglang{Rcpp} template functions \code{Rcpp::as} and
\code{Rcpp::wrap}, for converting \proglang{R} types to \octave
types and \emph{vice versa}.
Currently these templates are not exported, but will probably be in the future.

\section{Calling R functions from Octave}

This is currently under development.
Interested users can find this feature under the branch \code{feature/Rfun} in
the GitHub repository:

\url{https://github.com/renozao/RcppOctave/tree/feature/Rfun}

\section{Examples}
\subsection{Comparing implementations}

Comparing equivalent \R and \octave functions is as easy as comparing two \R
functions.
For example, one can compare the respective functions \code{svd} with the
following code, which defines a wrapper functions to format the output of
\octave \code{svd} function as \R (see \code{?svd} and \code{o\_help(svd)}):

\begin{knitrout}
\definecolor{shadecolor}{rgb}{0.969, 0.969, 0.969}\color{fgcolor}\begin{kframe}
\begin{alltt}
\hlstd{o_svd} \hlkwb{<-} \hlkwa{function}\hlstd{(}\hlkwc{x}\hlstd{) \{}
    \hlcom{# ask for the complete decomposition}
    \hlstd{res} \hlkwb{<-} \hlstd{.O}\hlopt{$}\hlkwd{svd}\hlstd{(x,} \hlkwc{argout} \hlstd{=} \hlkwd{c}\hlstd{(}\hlstr{"u"}\hlstd{,} \hlstr{"d"}\hlstd{,} \hlstr{"v"}\hlstd{))}
    \hlcom{# reformat/reorder result}
    \hlstd{res}\hlopt{$}\hlstd{d} \hlkwb{<-} \hlkwd{diag}\hlstd{(res}\hlopt{$}\hlstd{d)}
    \hlstd{res[}\hlkwd{c}\hlstd{(}\hlnum{2}\hlstd{,} \hlnum{1}\hlstd{,} \hlnum{3}\hlstd{)]}
\hlstd{\}}

\hlcom{# define random data}
\hlstd{X} \hlkwb{<-} \hlkwd{matrix}\hlstd{(}\hlkwd{runif}\hlstd{(}\hlnum{25}\hlstd{),} \hlnum{5}\hlstd{)}

\hlcom{# run SVD in R}
\hlstd{svd.R} \hlkwb{<-} \hlkwd{svd}\hlstd{(X)}
\hlcom{# run SVD in Octave}
\hlstd{svd.O} \hlkwb{<-} \hlkwd{o_svd}\hlstd{(X)}
\hlkwd{str}\hlstd{(svd.O)}
\end{alltt}
\begin{verbatim}
## List of 3
##  $ d: num [1:5] 3.1565 0.7806 0.3641 0.2775 0.0596
##  $ u: num [1:5, 1:5] -0.351 -0.474 -0.415 -0.397 -0.568 ...
##  $ v: num [1:5, 1:5] -0.391 -0.502 -0.447 -0.415 -0.471 ...
\end{verbatim}
\begin{alltt}
\hlcom{# check results}
\hlkwd{all.equal}\hlstd{(svd.R, svd.O)}
\end{alltt}
\begin{verbatim}
## [1] TRUE
\end{verbatim}
\begin{alltt}
\hlcom{# but not exactly identical}
\hlkwd{all.equal}\hlstd{(svd.R, svd.O,} \hlkwc{tol} \hlstd{=} \hlnum{10}\hlopt{^-}\hlnum{16}\hlstd{)}
\end{alltt}
\begin{verbatim}
## [1] "Component 2: Mean relative difference: 3.758e-16"
## [2] "Component 3: Mean relative difference: 2.931e-16"
\end{verbatim}
\end{kframe}
\end{knitrout}


\subsection{Random computations}

In order to ensure reproducibility of results and facilitate the comparability
of implementations between \R and \octave, \pkgname{RcppOctave} ships
a custom \octave module that redefine \octave standard random number generator
functions \code{rand}, \code{randn}, \code{rande} and \code{randg}, so that they
call \R corresponding functions \code{runif}, \code{rnorm}, \code{rexp} and
\code{rgamma}.
This module is loaded when the \Rpkg{RcppOctave} is itself loaded.
As a result, random computation -- that use these functions -- can be seeded in
both \octave and \R, using \R standard function \code{set.seed}.
This facilitates, in particular, the validation of ports of stochastic
algorithms (e.g. simulations, MCMC-based estimations):

\begin{knitrout}
\definecolor{shadecolor}{rgb}{0.969, 0.969, 0.969}\color{fgcolor}\begin{kframe}
\begin{alltt}
\hlstd{Rf} \hlkwb{<-} \hlkwa{function}\hlstd{()\{}
        \hlstd{x} \hlkwb{<-} \hlkwd{matrix}\hlstd{(}\hlkwd{runif}\hlstd{(}\hlnum{100}\hlstd{),} \hlnum{10}\hlstd{)}
        \hlstd{y} \hlkwb{<-} \hlkwd{matrix}\hlstd{(}\hlkwd{rnorm}\hlstd{(}\hlnum{100}\hlstd{),} \hlnum{10}\hlstd{)}
        \hlstd{(x} \hlopt{*} \hlstd{y)} \hlopt \hlstd{(x} \hlopt{/} \hlstd{y)}
\hlstd{\}}

\hlstd{Of} \hlkwb{<-} \hlstd{\{}
\hlcom{# define Octave function}
\hlkwd{o_source}\hlstd{(}\hlkwc{text}\hlstd{=}\hlstr{"
function [res] = test()
x = rand(10);
y = randn(10);
res = (x .* y) * (x ./ y);
end
"}\hlstd{)}
\hlcom{# return the function}
\hlstd{.O}\hlopt{$}\hlstd{test}
\hlstd{\}}

\hlcom{# run both computations with a common seed  }
\hlkwd{set.seed}\hlstd{(}\hlnum{1234}\hlstd{); res.R} \hlkwb{<-} \hlkwd{Rf}\hlstd{()}
\hlkwd{set.seed}\hlstd{(}\hlnum{1234}\hlstd{); res.O} \hlkwb{<-} \hlkwd{Of}\hlstd{()}
\hlcom{# compare results}
\hlkwd{identical}\hlstd{(res.R, res.O)}
\end{alltt}
\begin{verbatim}
## [1] TRUE
\end{verbatim}
\begin{alltt}
\hlcom{# not seeding the second computation would give different results}
\hlkwd{set.seed}\hlstd{(}\hlnum{1234}\hlstd{);}
\hlkwd{identical}\hlstd{(}\hlkwd{Rf}\hlstd{(),} \hlkwd{Of}\hlstd{())}
\end{alltt}
\begin{verbatim}
## [1] FALSE
\end{verbatim}
\end{kframe}
\end{knitrout}


\section{Known issues}
\label{sec:issues}

\begin{itemize}
 \item In \octave versions older than 3.6.1, the function \code{o\_ls} may not
 list user-defined functions. 
 This is due to the built-in \octave function \code{completion\_matches} that
 does not return them. The issue seems to have been fixed by \octave team at
 least in 3.6.1.
 \item The detection of output names by \code{.CallOctave} in \octave versions
 older than 3.4.1 does not work, meaning that \octave functions are always
 called with a single output variable.
 For obtaining more outputs, the user must specify argument \code{argout}
 accordingly.
 \item Redirection of Octave output sent to stdout and stderr on Windows does
 not work.
\end{itemize}

\section{News and changes}

{\scriptsize
\begin{verbatim}
********************
Changes in 0.14
********************
CHANGES
    o added a configure option --with-octave to specify the path to a non-standard
    Octave installation.
    
FIXES
    o Some changes have been made so that the package is compatible with Octave 3.8
    (reported by Bernard)
    o Installation was failing when Octave was compiled with hdf5-mpi support.
    
    
********************
Changes in 0.11
********************
NEW FEATURES
    o The package have successfully been installed on Windows machines, 
    although only basic functionalities have been tested (see README file).
    o Support for Mac have also started, although even more slightly, using
    the Octave version available from homebrew.

CHANGES
    o Moved all developments/bug reports/static docs to GitHub repository:
    https://github.com/renozao/RcppOctave/ 
    o Octave functions' stdout and stderr messages are now buffered by default, so 
    that it does not bypass R own i/o functions.
    All messages/warnings sent to stdout/stderr from Octave are displayed on 
    exiting the function call.
    o function .CallOctave gains a new argument buffer.std to enable/disable 
    stdout and/or stderr buffering (see ?.CallOctave).
    o Octave startup warnings (e.g. shadowing of core functions by Octave modules) 
    are not shown anymore.
    o Minor adaptations to pass new CRAN checks
    o The package now depends on R >= 3.0.0 to properly handle the knitr vignettes.
    o Errors are now more properly handled, thanks to hints found in this 
    old post by Romain François:
    http://lists.r-forge.r-project.org/pipermail/rcpp-devel/2010-May/000651.html

********************
Changes in 0.9.3
********************
CHANGES
    o Conditional use of function packageName: use the one from pkgmaker 
    in R <= 2.15.3, or the one exported by utils in R >= 3.0. 

\end{verbatim}
}

\section*{Session information}
{\small
\begin{knitrout}
\definecolor{shadecolor}{rgb}{0.969, 0.969, 0.969}\color{fgcolor}\begin{kframe}
\begin{verbatim}
R version 3.0.2 (2013-09-25)
Platform: x86_64-pc-linux-gnu (64-bit)

locale:
 [1] LC_CTYPE=en_US.UTF-8       LC_NUMERIC=C              
 [3] LC_TIME=C                  LC_COLLATE=en_US.UTF-8    
 [5] LC_MONETARY=en_US.UTF-8    LC_MESSAGES=en_US.UTF-8   
 [7] LC_PAPER=en_US.UTF-8       LC_NAME=C                 
 [9] LC_ADDRESS=C               LC_TELEPHONE=C            
[11] LC_MEASUREMENT=en_US.UTF-8 LC_IDENTIFICATION=C       

attached base packages:
[1] methods   stats     graphics  grDevices utils     datasets  base     

other attached packages:
[1] RcppOctave_0.14 pkgmaker_0.20   registry_0.2    Rcpp_0.11.0    
[5] knitr_1.5      

loaded via a namespace (and not attached):
[1] codetools_0.2-8 digest_0.6.4    evaluate_0.5.1  formatR_0.10   
[5] highr_0.3       stringr_0.6.2   tools_3.0.2     xtable_1.7-1   
\end{verbatim}
\end{kframe}
\end{knitrout}

}

\printbibliography[heading=bibintoc]

\end{document}
